\documentclass{beamer}

\usepackage{graphicx}
\usepackage[caption=false]{subfig}
\usepackage{amsmath}

\title{Deep Learning in NLP: A teaser trailer\\ \small{TaCoS 2018}}
\author{Jens Johannsmeier}
\date{October 10, 2016}


\begin{document}

\begin{frame}
\titlepage
\end{frame}


\begin{frame}{Why Machine Learning?}
\begin{figure}
\includegraphics[width=\textwidth]{pictures/simple/easy.png}
\end{figure}
\end{frame}


\begin{frame}{Why Machine Learning?}
\begin{figure}
\includegraphics[width=\textwidth]{pictures/simple/easy_solved.png}
\end{figure}
\end{frame}


\begin{frame}{A more realistic problem}
\begin{figure}
\includegraphics[scale=0.5]{pictures/speech/wave_yes.png}
\end{figure}
What word is being spoken here?
\end{frame}


\begin{frame}{A more realistic problem}
\begin{figure}
\subfloat{\includegraphics[scale=0.37]{pictures/speech/wave_yes.png}}
\subfloat{\includegraphics[scale=0.37]{pictures/speech/stft_yes.png}}
\end{figure}
bed, bird, cat, dog, down, eight, five, four, go, happy, house, left, marvin, nine, no, off, on, one, right, seven, sheila, six, stop, three, tree, two, up, wow, yes, zero?
\end{frame}


\begin{frame}{Another more realistic problem}
``This is the worst movie I have ever seen.'' \\
``This is a good movie.'' \\
``This is not good.'' \bigskip

Are these positive or negative statements?
\end{frame}


\begin{frame}{Nonlinear problems}
\begin{figure}
\includegraphics[width=\textwidth]{pictures/simple/circles.png}
\end{figure}
\end{frame}


\begin{frame}{Transforming data}
\begin{figure}
\includegraphics[width=\textwidth]{pictures/simple/circles_polar.png}
\end{figure}
\end{frame}


\begin{frame}{Another nonlinear problem}
\begin{figure}
\includegraphics[width=\textwidth]{pictures/simple/spiral.png}
\end{figure}
\end{frame}


\begin{frame}{Polar coordinates...?}
\begin{figure}
\includegraphics[width=\textwidth]{pictures/simple/spiral_polar.png}
\end{figure}
\end{frame}


\begin{frame}{Machine Learning as Pattern Matching}
\begin{figure}
\includegraphics[scale=0.5]{pictures/simple/easy_solved.png}
\end{figure}
$c_x = 2.99, c_y = 1.37$
\end{frame}


\begin{frame}{Machine Learning as Pattern Matching}
\begin{figure}
\includegraphics[scale=0.5]{pictures/simple/circles_polar.png}
\end{figure}
$c_x = 22.59, c_y = -0.21$
\end{frame}


\begin{frame}{Machine Learning as Pattern Matching}
Examples from sentiment and maybe audio.\\
Show limitations of simple features + linear models (no interaction).
\end{frame}


\begin{frame}{Non-linear Feature Composition}
Real or constructed example of XOR-like problem (e.g.~not good/bad).
\end{frame}


\begin{frame}{Multilayer Perceptrons}
Large-scale examples for sentiment analysis and speech recognition, with examples of learned features!
Show inadequacy of e.g. bag-of-words representations to motivate more complex structures (RNNs).
\end{frame}


\begin{frame}{Recurrent Neural Networks}
Schemes of recurrent/unrolled graphs.
\end{frame}


\begin{frame}{Sentiment Analysis with RNNs}
Show hopefully good results and how RNN handles ``not good''.\\
Generally analyze connections etc.
\end{frame}


\begin{frame}{Convolutional Neural Networks}
Give idea of same unit repeated many times over space; local connections.\\
Visual explanation of filters.
\end{frame}


\begin{frame}{Speech recognition with CNNs}
Show patterns learned by CNN.
\end{frame}

\end{document}